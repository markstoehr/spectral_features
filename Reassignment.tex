%% LyX 2.0.2 created this file.  For more info, see http://www.lyx.org/.
%% Do not edit unless you really know what you are doing.
\documentclass[english]{article}
\usepackage[T1]{fontenc}
\usepackage[latin9]{inputenc}
\usepackage{listings}
\usepackage{graphicx}
\usepackage{esint}
\usepackage{babel}
\usepackage{fullpage}
\usepackage{hyperref}
\usepackage{amsmath}
\usepackage{amssymb}
\usepackage{subcaption}
\usepackage{tikz}
\usepackage{verbatim}

%\theoremstyle{plain}% default
\newtheorem{thm}{Theorem}[section]
\newtheorem{lem}[thm]{Lemma}
\providecommand*{\lemautorefname}{Lemma}
\newtheorem{prop}[thm]{Proposition}
\newtheorem{cor}[thm]{Corollary}
%\theoremstyle{definition}
\newtheorem{defn}{Definition}[section]
\newtheorem{conj}{Conjecture}[section]
\newtheorem{exmp}{Example}[section]
%\theoremstyle{remark}
\newtheorem{rem}{Remark}
\newtheorem{note}{Note}
\newtheorem{case}{Case}

\usepackage[ruled]{algorithm2e}
\renewcommand{\algorithmcfname}{ALGORITHM}
\SetAlFnt{\small}
\SetAlCapFnt{\small}
\SetAlCapNameFnt{\small}


\title{Reassignment Fixed Points}
\author{Mark Stoehr}
\date{\today}


\begin{document}

\maketitle
\tableofcontents

\section{Introduction}

Time Frequency reassignment is an algorithm for improving spectrogram readability \cite{auger03}.  We wish to
contribute to an alternative use of the reassignment computation where we use it for spectral peak detection
rather than spectral transformation.  One of the standard interpretations of reassignment is that it finds
the ``center-of-mass'' of a time-frequency distribution (commonly the Wigner-Ville) and it corrects for the
distortion created by a time-frequency computation.  We will instead interpret it as a mode-seeking algorithm
for the Wigner-Ville distribution and use it to give a binary mode-map of the time-frequency data.

Throughout the paper $x[t]$ will denote a discrete-time amplitude signal and $x(t)$ will denote
a continuous-time amplitude signal at time $t$. We presume that all observations are on the interval $[0,T]$.
Generally $F$ will either denote the maximum considered frequency or the number of frequency bins $T$ may
also denote the number of time bins.

\section{Reassignment Algorithms}

Given a Wigner-Ville distribution $W(t,f)$ for the signal $x(t)$ and a kernel $h\left(\frac{|t-\tau|}{\sigma}\right)$ we 
may write the reassigned times $(\hat{t},\hat{f})$ for $(t,f)$ as \cite{auger03}
\begin{align}
\hat{t} &=  t -\frac{\int\int \tau W(t-\tau,f-\omega) h\left(\frac{|\tau|}{\sigma_1}\right)h\left(\frac{|\omega|}{\sigma_2}\right)d\tau d\omega}{
 \int\int W(t-\tau,f-\omega) h\left(\frac{|\tau|}{\sigma_1}\right)h\left(\frac{|\omega|}{\sigma_2}\right)d\tau d\omega}\\
\hat{f} &=  f -\frac{\int\int \omega W(t-\tau,f-\omega) h\left(\frac{|\tau|}{\sigma_1}\right)h\left(\frac{|\omega|}{\sigma_2}\right)d\tau d\omega}{
 \int\int W(t-\tau,f-\omega) h\left(\frac{|\tau|}{\sigma_1}\right)h\left(\frac{|\omega|}{\sigma_2}\right)d\tau d\omega}.
\end{align}
and that we may more succinctly compute these in the case of a spectrogram as
\begin{align}
\hat{t}(t,w) &= t - \Re\left\{\frac{X(t,f; \mathcal{T}h) X^*(t,f; h)}{|X(t,f; h)|^2 }\right\}\\
\hat{f}(t,w) &= f + \Im\left\{\frac{X(t,f; \mathcal{D}h) X^*(t,f; h)}{|X(t,f; h)|^2 }\right\}
\end{align}
where $X(t,f; h)$ is the spectrogram computed with window $h$ at time-frequency location $t,f$, $\mathcal{D}h(t)=\frac{d}{dt}h(t)$, and $\mathcal{T}h(t)=th(t)$.
Usually the reassignment estimates are used to construct a reassigned time-frequency distribution $R(t,f)$ where
\begin{equation}
R(t,f) = \int\int W(\tau,\omega) \delta(\tau-\hat{t}(t,f))\delta(\omega -\hat{f}(t,f))d\tau d\omega.
\end{equation}
However, we propose to find the modes of the reassigned spectrogram, which are those points that are not mapped to other points.  Namely reassignment
is essentialy just a function from the data to itself and we expect to find fixed points along ridges and on impulsive noise.  We present a summary of a basic fixed-point finding algorithm
in \autoref{alg:basic-fixed-points}.
\begin{algorithm}[t]
\SetAlgoNoLine
\KwIn{Reassignment operators $\hat{t}\colon [T]\times [F]\to [T]$, $\hat{f}\colon [T]\times [F]\to [F]$, a stack $U$, quantization factor $q$ for frequency, quantization factor $r$ for time}
\KwOut{Fixed Point Map $M\in ([T]\times[F])^{T\times F}$}
$\forall(t,f)\;M(t,f) \leftarrow (-1,-1)$\;

\For{$t=1,2,\ldots,T$}{
  \For{$f=1,2,\ldots,F$}{
    NotConverged $\leftarrow$ True\;
    \While{ NotConverged}{

         \uIf{$M(t,f)\neq (-1,-1)$ }{
             NotConverged $\leftarrow$ False\;
             
           }\Else{
           Push $(t,f)$ onto $U$\;
           $(t,f)\leftarrow (\hat{t}(t,f),\hat{f}(t,f))$\;
           $M(t,f)\leftarrow  (\lfloor t/r+\frac{1}{2}\rfloor,\lfloor f/q +\frac{1}{2}\rfloor)$\;
           
         }
      
    \While{$U$ not empty}{
      $(t',f')\leftarrow $ pop $ U$\;
      $M(t',f') = M(t,f)$\;
      
      }
    }
}}
  
\caption{Find Fixed Points}
\label{alg:basic-fixed-points}
\end{algorithm}
The basic idea is to start from each location of the time-frequency array and go in the direction
indicated by the function until you hit a location that you have already visited which is
declared the mode.  Each
node of the time-frequency representation is then marked by the end point of the path the
are on.  We include quantization factors as we may only want a coarse sampling of the time frequency plane for locations.  The number of locations in the fixed point map $M$ will generally be
much smaller than the total number of locations and thus we get a binary map indicating 
which locations are fixed points.  The key loop invariant that the algorithm maintains is that
by the end of the main while block every entry of $M$ either points to $(-1,-1)$ or to a mode.

\subsection{Empirical Evaluation of Simple Fixed Point Finder}

In our empirical evaluation we are primarily testing whether the fixed points found
by the algorithm correspond to the type of strucures we think are important in the
spectrogram.

\bibliographystyle{plain}
\bibliography{../bibliography}
\end{document}

%% This file is part of spectral_features.

%%     spectral_features is free software: you can redistribute it and/or modify
%%     it under the terms of the GNU General Public License as published by
%%     the Free Software Foundation, either version 3 of the License, or
%%     (at your option) any later version.

%%     spectral_features is distributed in the hope that it will be useful,
%%     but WITHOUT ANY WARRANTY; without even the implied warranty of
%%     MERCHANTABILITY or FITNESS FOR A PARTICULAR PURPOSE.  See the
%%     GNU General Public License for more details.

%%     You should have received a copy of the GNU General Public License
%%     along with spectral_features.  If not, see <http://www.gnu.org/licenses/>.


\documentclass[english]{article}
\usepackage[T1]{fontenc}
\usepackage[latin9]{inputenc}
\usepackage{listings}
\usepackage{graphicx}
\usepackage{esint}
\usepackage{babel}
\usepackage{fullpage}
\usepackage{hyperref}
\usepackage{amsmath}
\usepackage{amssymb}
\usepackage{subcaption}
\usepackage{tikz}
\usepackage{verbatim}

%\theoremstyle{plain}% default
\newtheorem{thm}{Theorem}[section]
\newtheorem{lem}[thm]{Lemma}
\providecommand*{\lemautorefname}{Lemma}
\newtheorem{prop}[thm]{Proposition}
\newtheorem{cor}[thm]{Corollary}
%\theoremstyle{definition}
\newtheorem{defn}{Definition}[section]
\newtheorem{conj}{Conjecture}[section]
\newtheorem{exmp}{Example}[section]
%\theoremstyle{remark}
\newtheorem{rem}{Remark}
\newtheorem{note}{Note}
\newtheorem{case}{Case}

\usepackage[ruled]{algorithm2e}
\renewcommand{\algorithmcfname}{ALGORITHM}
\SetAlFnt{\small}
\SetAlCapFnt{\small}
\SetAlCapNameFnt{\small}


\title{Fast Mean-Shift on the Wigner-Ville Distribution}
\author{Mark Stoehr}
\date{\today}


\begin{document}

\maketitle
\tableofcontents

\section{Introduction}

In this paper we introduce a framework for speech recognition using
time-frequency ridges and modes.  
\begin{enumerate}
\item Problem: Spectrogram has poor localization and not cross terms
\item Problem: Wigner-Villed has good localization but cross terms
\item Why are these a problem: localization is important for distinguishing speech sounds
\item cross terms mean that its hard to tell what corresponds to a component and what is a phantom
\item Partial solution: Reassignment moves energy to center-of-gravity--produces better localization
\item Problem: Still don't have a map of the modes
\item In reality we want a binary feature map that indicates the presence of a certain type of component or its non-presence
\item Idea: Mean-shift proceduces that kind of map
\item Problem: Mean-shift is slow and computationally annoying
\item Idea: Reassignment (under certain settings) is fast and is essentially one-step of mean-shift done at a dense grid of time-frequency locations
\item Idea: Use the reassignment vector-field to run mean-shift until it converges at all locations and output the resultant binary map
\end{enumerate}

Similar ideas to this were considered in \cite{tfsnakes,ozertem11} where the authors specifically drew the connection
between mean-shift and reassignment. However, they did not use reassignment algorithms to perform mean-shift on the WVD.


Acoustic phoneticians largely theorize that much of speech can described in
terms of formants, formant transitions, harmonics, frication,
and bursts.  In the language of time-frequency these are harmonic components,
chirps, and transients.  While the spectrogram remains dominant, researchers in
time-frequency have shown that the spectrogram performs poorly at localizing
harmonics, chirps, and transients.  The time-frequency representation that
localizes these phenomena optimally is the Wigner-Ville distribution (WVD), however
the WVD suffers from the problem that if we compute it on a signal with multiple
components, such as speech, in addition to perfectly localizing the components
there are cross-terms that are generated that are potentially negative and hard
to interpret.  Much work has gone into finding methods of smoothing the WVD
to eliminate the effects of these cross-terms and it has been shown that the 
spectrogram is a particular type of smoothing of the WVD.



  The purpose of these representations is to capture
time-varying frequency information, or, more generally an estimate
of the Fourier transform of the
autocovariance function.  In speech recognition these representations are important
because they convey information about the cues used in speech recognition.  
Acoustic phoneticians \cite{stevens98} generally describe speech in terms
of formants, bursts, and other features that correspond to modes and ridges in
a time-frequency distribution (generally the spectrogram). With a few exceptions
researchers have not explicitly set out to find these essentially diffeomorphic
characteristics of time-frequency representations or developed fast algorithms to
find them.  A prominent iterative algorithm, mean-shift, is very good at finding these
diffeomorphic structures, but its run time is prohibitively expensive in
a time-frequency context.  However, it has been noted that a single iteration of mean-shift
correspond to a standard time-frequency algorithm known as the reassignment method.
In this paper, we complete this connection by using a fast algorithm to run mean-shift
to convergence in the time-frequency plane giving us a practical mode-based time-frequency
representation.

Throughout the paper $x[t]$ will denote a discrete-time amplitude signal and $x(t)$ will denote
a continuous-time amplitude signal at time $t$. We assume that in either case the signal has zero mean.
 We presume that all observations are on the interval $[0,T]$.
Generally $F$ will either denote the maximum considered frequency or the number of frequency bins $T$ may
also denote the number of time bins. Except where otherwise noted, the domain of an integral $\int$ is $(-\infty,\infty)$.

\section{Mean-Shift}

We present a special case of the mean-shift algorithm that will be adapted for time-frequency purposes.
Given samples $\{x_n\}_{n=1}^N\subset \mathbb{R}^D$ and a normalized, nonnegative, symmetric function $K\colon \mathbb{R}^D\to\mathbb{R}_+$,
a standard estimate for the  probability density of $\{x_n\}_{n=1}^N$ is the kernel density estimate
\begin{equation}
\hat{\mathbb{P}}(x ; k) = \sum_{n=1}^N K(x-x_n).
\end{equation}
A common choice for $K$ is the Gaussian kernel $K(x-y)=\exp\left(-\frac{\|x-y\|^2}{2\sigma^2}\right)$ where
$\sigma$ is called the bandwidth parameter.  Mean-shift is an algorithm which given a point $y\in\mathbb{R}^D$
iteratively hill-climbs
$\hat{\mathbb{P}}(x ; \sigma)$ from $y$ so that it conveges on a mode $y_*$, so that all the points close to
a mode will converge to it \cite{fukunaga75,chengms95,Comaniciu02}. The algorithm is presented in the 
Gaussian kernel case in \autoref{alg:gauss-mean-shift}.
\begin{algorithm}[t]
\SetAlgoNoLine
\KwIn{Samples $\{x_n\}_{n=1}^N\subset\mathbb{R}^D$, query point $y\in\mathbb{R}^D$, bandwidth-parameter $\sigma > 0$, tolerance $\epsilon > 0$}
\KwOut{Estimated pdf mode $\hat{y}$ near $y$}

$y_0\leftarrow y$\;
$i\leftarrow 0$\;
\While{$\frac{\|y_{i}-y_{i-1}\|}{\|y\|} > \epsilon$}{
$y_{i+1} = \frac{\sum_{n=1}^N x_n \exp\left\{-\frac{1}{2\sigma^2}\|y_i-x_n\|^2\right\}}{\sum_{n=1}^N \exp\left\{-\frac{1}{2\sigma^2}\|y_i-x_n\|^2\right\}}$\;
$i\left\leftarrow i+1$\;
}
\Return  $y_i$\;
\caption{Gaussian Kernel Mean Shift}
\label{alg:gauss-mean-shift}
\end{algorithm}
The algorithm is iterative and in each round your estimate $y$ is updated to $y'$ where
\begin{equation}\label{eq:mean-shift-step}
y' = \frac{\sum_{n=1}^N x_n \exp\left\{-\frac{1}{2\sigma^2}\|y-x_n\|^2\right\}}{\sum_{n=1}^N \exp\left\{-\frac{1}{2\sigma^2}\|y-x_n\|^2\right\}}.
\end{equation}

\subsection{Weighted Mean-Shift}

In this paper we will investigate a weighted variant of mean shift where the mean shift step in \autoref{eq:mean-shift-step}
is instead
\begin{equation}\label{eq:weighted-mean-shift-step}
y' = \frac{\sum_{n=1}^N x_n w_n \exp\left\{-\frac{1}{2\sigma^2}\|y-x_n\|^2\right\}}{\sum_{n=1}^N w_n\exp\left\{-\frac{1}{2\sigma^2}\|y-x_n\|^2\right\}}
\end{equation}
where the weights $\{w_n\}_{n=1}^N$ are nonnegative, in particular they will have a time-frequency meaning.
We recall that Comaniciu showed in \cite{Comaniciu02} that mean-shift enjoys the standard mode convergence properties with
weighted samples we conjecture that the modes will be the modes of the weighted KDE estimate
\begin{equation}
\mathbb{F}(y; \sigma) = \sum_{n=1}^N w_n \exp\left\{-\frac{1}{2\sigma^2}\|y-x_n\|^2\right\}.
\end{equation}

\section{Time-Frequency}

Quadratic time-frequency distributions of a signal $x(t)$ may be written \cite[eq. 5]{andrieux87}
\begin{equation}\label{eq:andrieux-general}
F_x(t,\omega ; K)=\int\int K(t-t',\omega-\omega')W_x(t',\omega')\;dt'\;d\omega'
\end{equation}
where  $W_x$ is the Wigner-Ville distribution (WVD) is the Fourier transform of the empirical
autocovariance function (given that the first moment of $x$ is zero)
\begin{equation}\label{eq:wigner-ville-distribution}
  W_x(t,\omega) = \int x(t+\frac{1}{2}\tau)x^*(t-\frac{1}{2}\tau)e^{-i\omega\tau}\;d\tau,
\end{equation} 
and $K$ is a general smoothing kernel (usually a low-pass filter). As shown in \cite{andrieux87} by
letting (!!! NOT SURE ABOUT THIS STEP -- JUST WHAT AUTHOR CLAIMS !!!) (ACTUALLY ITS PROBABLY THE CONVOLUTION
THEOREM BUT I NEED TO WORK OUT THE DETAILS)
\begin{equation}
K(t,\omega) = \frac{1}{(2\pi)^2} \int\int \phi(\tau,\theta)\exp{i(\tau t +\omega\theta)}d\tau\;d\omega
\end{equation}
\autoref{eq:andrieux-general} can be shown to encompass all of  Cohen's class (which includes every
quadratic time-frequency representation) \cite[p. 136]{cohen95}:
\begin{equation}\label{eq:cohen-general}
C_x(t,\omega ; \phi)=\frac{1}{4\pi^2} \int\int\int x(u-\frac{1}{2}\tau)x(u+\frac{1}{2}\tau)\phi(\theta,\tau)e^{-i\theta t-i\tau\omega+i\theta u}\;du\;d\tau\;d\theta.
\end{equation}
As such quadratic time-frequency distributions are simply a smoothed
Fourier transform of the empirical autocovariance function, so we may concentrate on 
properties of the WVD without any loss of generality. The WVD, as shown in \cite[ch. 2]{cohen95}
perfectly localizes
\begin{enumerate}
\item harmonics
\item chirps
\item transients
\end{enumerate}
and that in many simple signal mixtures these will correspond to the modes of WVD so it is natural to ask whether we can find
those modes using mean-shift.
We let $D=2$ and $\{(t_n,\omega_n)\}_{n=1}^N\subset\mathbb{R}^D$ where each $(t_n,\omega_n)$ is a time-frequency coordinate,
$\{(t_n,\omega_n)\}_{n=1}^N$ be a dense lattice of samples over some interval $[0,T]\times [0,\Omega]$, and
$\{w_n\}_{n=1}^N$ be weights where $w_n=W_x(t_n,\omega_n )$. Then we have a mean-shift iteration on the WVD, but
we run into a problem because the conditions for convergence given in \cite{Comaniciu02} require that the weights
be nonnegative.

\subsection{Hermite Smoothed Wigner-Ville Distribution and Spectrograms}

\begin{enumerate}
\item Wigner-Ville distribution of a Hermite function (physicist Hermite polynomial modulated by a Gaussian) is just a Hermite function
\item Spectrogram is just the Wigner-Ville Distribution of the signal convolved with the Wigner-Ville distribution of the window function
\item So Hermite-windowed spectrograms are just Hermite function smoothed (the special case of Gaussian windows also holds true)
\item Also spectrogram is always nonnegative so weighted mean-shift with appropriately chosen kernels weighted by the WVD will have the
requisite non-negativity
\end{enumerate}

 Under these conditions
we conjecture that weighted mean-shift in \autoref{eq:weighted-mean-shift-step}
will converge to modes of the WVD corresponding to single signal components.  We also
note that we are working in the continuous-time case so the weighted mean-shift step is
\begin{equation}\label{continuous-wvd-mean-shift}
(t',\omega') =\left(\frac{\int \int (t-u)\phi(u,\nu) W_x(t-u,\omega-\nu)\; du\;d\nu}{\int \int \phi(u,\nu) W_x(t-u,\omega-\nu)\; du\;d\nu},
\frac{\int \int (\omega-\nu)\phi(u,\nu) W_x(t-u,\omega-\nu)\; du\;d\nu}{\int \int \phi(u,\nu) W_x(t-u,\omega-\nu)\; du\;d\nu}
\right).
\end{equation}

A more general result establishes the connection to smoothing:
\begin{thm}
Let $\mathbb{X}$ denote the set of signal, i.e. functions $\mathbb{C}\to\mathbb{C}$.
The set of spectrograms $\{ S_x(t,\omega; h\}_{h\in\mathcal{H},x\in\mathbb{X}}$ where
\begin{equation}
S_x(t,\omega ; h) = \left|\int x(t-\tau)h(\tau)e^{-i\omega\tau}\;d\tau\right|^2
\end{equation}
and $\mathcal{H}$ is the set of window functions
is precisely the set of Wigner-Ville distributions smoothed by a
stationary positive semi-definite kernel
$\{ F_x(t,\omega ; K)\}_{K\in\mathcal{K}_{stationary},x\in\mathbb{X}}$
where 
\begin{equation}
F_x(t,\omega ; K) = \int\int K(t-\tau,\omega-\nu) W_x(\tau,\omega)\;d\tau\;d\omega.
\end{equation}
\end{thm}

We can then prove a general sufficient condition for convergence of mean-shift when
working with a kernel smoothed WVD:
\begin{thm}
Whenever $K$ is a stationary positive semi-definite kernel then mean-shift using that kernel
converges and finds modes. 
\begin{equation}

\end{equation}
\end{thm}

\section{Time-Frequency Reassignment}

Having worked out some of the theory
we are then in a position to consider a practical proposition about running
mean shift on the signal
\begin{thm}
Whenever $K$ is a stationary positive semi-definite kernel with time support-length $\Omega$
 then mean-shift using that kernel
over the discrete Wigner-Ville distribution (DWVD) on a discrete sampled signal $x[t]$ with
$T$ points runs in time $O(T\Omega\log \Omega)$.
\end{thm}
The theorem is proven by showing that the algorithm can be broken into two parts:
\begin{enumerate}
\item Mean-shift vector field computation over all coordinates $\{(t_n,\omega_n)\}_{n=1}^N$ (where $N=O(T\Omega)$) which is
done in time $O(T\Omega\log\Omega)$.
\item Assign a mode to each point in the vector field by tracing it out--which has time complexity $O(T\Omega)$.
\end{enumerate}
The first item is shown via this theorem applied to the discrete domain:
\begin{thm}
Let $\phi(\tau-\tau',\nu-\nu')$ be a stationary positive semi-definite kernel, $x(t)$ be a signal, and $\mathcal{T}$ defined so
 $\mathcal{T}_tf(t,\cdot)=tf(t,\cdot)$ then there exists a window function $h$ such that
\begin{equation}
\phi(t,\omega) = \int h(t+\frac{1}{2}\tau)h^*(t-\frac{1}{2}\tau)e^{-i\omega\tau}\;d\tau
\end{equation}
so
\begin{equation}
(\mathcal{T}\phi * W_x)(t,\omega) = R_x(t,\omega;h) R_x(t,\omega; \mathcal{T}h)
\end{equation}
and
\begin{equation}
(\mathcal{D}_t\phi * W_x)(t,\omega) = R_x(t,\omega;h) R_x(t,\omega; \mathcal{D}h)
\end{equation}
where $\mathcal{D}_t=\frac{\partial}{\partial t}$ and
\begin{equation}
R_x(t,\omega; h) = \int x(t-\tau)h(\tau)e^{-i\tau\omega}\; d\tau
\end{equation}
is the short-time Fourier transfrom (STFT) with window $h$.
\end{thm}
There is a continuous time version and discrete time version of the above theorem. Since we 
are working on discrete-time signals we will prove the discrete
time version.  An important lemma that we use is
\begin{lem}
Let $W_x[t,\omega; h,g]$ be a function on $(t,\omega)$ with window functions $h$ and $g$ so that
\begin{equation}
W_x[t,\omega; h,g] = \sum_{\tau=0}^{T-1}x(t+\frac{1}{2}\tau)
\end{equation} 
\end{lem}

We now give fast algorithms to compute the quantities:
\begin{align}
&\int\int (t-u) \phi(u,\nu) W_x(t-u,\omega - \nu)\; du\;d\nu \\
&\int\int (\omega-\nu)\phi(u,\nu) W_x(t-u,\omega-\nu)\; du\;d\nu
\end{align}
based on the following theorem which is known in the literature \cite{auger95,cohen95} but
we will find it useful to give a complete proof

which are the two components of the numerators of the quantities in \autoref{continuous-wvd-mean-shift}.
In the case where we are working with Gaussian window functions for $\phi(u,\nu)$
the smoothed WVD is a spectrogram (or sum of spectrograms) so
\begin{align*}
&\int\int \nu \phi(u,\nu) W_x(t-u,\omega-\nu)\; du\;d\nu \\
=& \int\int \nu \kappa \exp\left(-\frac{1}{2\alpha}u^2\right) \exp\left(-\frac{1}{2\beta}\nu^2\right) \int x(t-u+\frac{1}{2}\tau)x(t-u-\frac{1}{2}\tau)e^{-i\tau(\omega-\nu)}\;d\tau\; du\;d\nu\\
=& \int \kappa\beta \exp\left(-\frac{1}{2\alpha}u^2\right) \int \frac{\nu}{\beta} \exp\left(-\frac{1}{2\beta}\nu^2\right) \int x(t-u+\frac{1}{2}\tau)x(t-u-\frac{1}{2}\tau)e^{-i\tau(\omega-\nu)}\;d\tau\;d\nu\; du
\end{align*}
The rest of the derivation uses the convolution theorem and the fact that Hermite functions are eigenfunctions
of the Fourier transform.

Given a signal $x(t)$ the general class of time-frequency representations (which includes the spectrogram among
others) \cite{cohen95} may be
written
\begin{equation}
C(t,\omega)
\end{equation}

\section{Reassignment Algorithms}

Given a Wigner-Ville distribution $W(t,f)$ for the signal $x(t)$ and a kernel $h\left(\frac{|t-\tau|}{\sigma}\right)$ we 
may write the reassigned times $(\hat{t},\hat{f})$ for $(t,f)$ as \cite{auger03}
\begin{align}
\hat{t} &=  t -\frac{\int\int \tau W(t-\tau,f-\omega) h\left(\frac{|\tau|}{\sigma_1}\right)h\left(\frac{|\omega|}{\sigma_2}\right)d\tau d\omega}{
 \int\int W(t-\tau,f-\omega) h\left(\frac{|\tau|}{\sigma_1}\right)h\left(\frac{|\omega|}{\sigma_2}\right)d\tau d\omega}\\
\hat{f} &=  f -\frac{\int\int \omega W(t-\tau,f-\omega) h\left(\frac{|\tau|}{\sigma_1}\right)h\left(\frac{|\omega|}{\sigma_2}\right)d\tau d\omega}{
 \int\int W(t-\tau,f-\omega) h\left(\frac{|\tau|}{\sigma_1}\right)h\left(\frac{|\omega|}{\sigma_2}\right)d\tau d\omega}.
\end{align}
and that we may more succinctly compute these in the case of a spectrogram as
\begin{align}
\hat{t}(t,w) &= t - \Re\left\{\frac{X(t,f; \mathcal{T}h) X^*(t,f; h)}{|X(t,f; h)|^2 }\right\}\\
\hat{f}(t,w) &= f + \Im\left\{\frac{X(t,f; \mathcal{D}h) X^*(t,f; h)}{|X(t,f; h)|^2 }\right\}
\end{align}
where $X(t,f; h)$ is the spectrogram computed with window $h$ at time-frequency location $t,f$, $\mathcal{D}h(t)=\frac{d}{dt}h(t)$, and $\mathcal{T}h(t)=th(t)$.
Usually the reassignment estimates are used to construct a reassigned time-frequency distribution $R(t,f)$ where
\begin{equation}
R(t,f) = \int\int W(\tau,\omega) \delta(\tau-\hat{t}(t,f))\delta(\omega -\hat{f}(t,f))d\tau d\omega.
\end{equation}

\subsection{Discrete Mean Shift}

Suppose we are given are a given a discrete signal $x(t)$ defined on $\mathbb{Z}$ but with support on $t=0,1,\ldots,T-1$  and we compute a discrete short-time Fourier transform
$X(t,f; h)$ with window $h(t)$ with support $-N/2,1-N/2,\ldots,N/2-1$
\begin{align}
X(t,f; h) =& \sum_{\tau=-N/2}^{N/2-1} X(t+\tau)h(\tau)e^{- i 2\pi\frac{f}{N}(\tau + N/2)}\\
=& \sum_{\tau=-N/2}^{N/2-1} X(t+\tau)h(\tau)e^{-i 2\pi\frac{f}{N} (\tau + N/2)}\\
=& \sum_{\tau=-\infty}^{\infty} X(\tau)h(\tau-t)e^{-i 2\pi\frac{f}{N}(\tau-t + N/2)}
\end{align}
The spectrogram $|X(t,f; h)|^2=X(t,f; h)X^*(t,f;h)$ may then be written as
\begin{align}
X(t,f; h)X^*(t,f; h) =& \sum_{\tau=-\infty}^{\infty}\sum_{\tau'=-\infty}^{\infty} X(\tau)X(\tau')h(\tau-t)h(\tau'-t)e^{-i 2\pi\frac{f}{N}(\tau-\tau')}.
\end{align}
Then we have that
\begin{equation}
tX(t,f; h)X^*(t,f; h) =& \sum_{\tau=-\infty}^{\infty}\sum_{\tau'=-\infty}^{\infty} X(\tau)X(\tau')h(\tau-t)h(\tau'-t)e^{-i 2\pi\frac{f}{N}(\tau-\tau')}
\end{equation}
This means that each mean-shift step
is
\begin{align}
\hat{t}(t,f) =& \frac{\sum_{u=-\infty}^\infty (t-u) |X(t-u,f;h)|^2}{ \sum_{u=-\infty}^\infty |X(t-u,f;h)|^2}\\
 =& t- \frac{\sum_{u=-\infty}^\infty u \sum_{\tau=-N/2}^{N/2-1}\sum_{\tau'=-N/2}^{N/2-1} X((t-u)+\tau)X((t-u)+\tau')h(-\tau)h(-\tau')e^{-i 2\pi\frac{f}{N}(\tau-\tau')}}{\sum_{u=-\infty}^\infty \sum_{\tau=-N/2}^{N/2-1}\sum_{\tau'=-N/2}^{N/2-1} X((t-u)+\tau)X((t-u)+\tau')h(-\tau)h(-\tau')e^{-i 2\pi\frac{f}{N}(\tau-\tau')}}
\end{align}

\subsection{The Discrete Case}

In practical work we have discrete time signals use discrete-time algorithms.
A discrete time spectrogram $X$ with window $h$ is computed as
\begin{align*}
X(t,f; h) =& \frac{1}{2}\left|\sum_{\tau=-N/2}^{N/2-1} X(t-\tau)h(\tau)e^{i 2\pi\frac{f}{N}(\tau + N/2)}\right|^2\\
=& \frac{1}{2}\left(\sum_{\tau=-N/2}^{N/2-1} X(t-\tau)h(\tau)e^{i 2\pi\frac{f}{N}(\tau + N/2)}\right)\left(\sum_{\tau=-N/2}^{N/2-1} X(t-\tau)h(\tau)e^{i 2\pi\frac{f}{N}(\tau + N/2)}\right)^*\\
=&  \frac{1}{2}\sum_{\tau=-N/2}^{N/2-1}\sum_{\tau'=-N/2}^{N/2-1} X(t-\tau)X(t-\tau')h(\tau)h(\tau')e^{i 2\pi\frac{f}{N}(\tau -\tau')} \\
=&  \sum_{k=0}^{N/2-1}\sum_{\tau=0}^k X(t-\frac{N}{2}+1+\frac{k}{2}-\tau)X(t-\frac{N}{2}+1+\frac{k}{2}-\tau)
\end{align}
which shows the relationship to the Wigner-Ville Distribution.

\subsection{Discrete Mean Shift}

We then show that 

However, we propose to find the modes of the reassigned spectrogram, which are those points that are not mapped to other points.  Namely reassignment
is essentialy just a function from the data to itself and we expect to find fixed points along ridges and on impulsive noise.  We present a summary of a basic fixed-point finding algorithm
in \autoref{alg:basic-fixed-points}.
\begin{algorithm}[t]
\SetAlgoNoLine
\KwIn{Reassignment operators $\hat{t}\colon [T]\times [F]\to [T]$, $\hat{f}\colon [T]\times [F]\to [F]$, a stack $U$, quantization factor $q$ for frequency, quantization factor $r$ for time}
\KwOut{Fixed Point Map $M\in ([T]\times[F])^{T\times F}$}
$\forall(t,f)\;M(t,f) \leftarrow (-1,-1)$\;

\For{$t=1,2,\ldots,T$}{
  \For{$f=1,2,\ldots,F$}{
    NotConverged $\leftarrow$ True\;
    \While{ NotConverged}{

         \uIf{$M(t,f)\neq (-1,-1)$ }{
             NotConverged $\leftarrow$ False\;
             
           }\Else{
           Push $(t,f)$ onto $U$\;
           $(t,f)\leftarrow (\hat{t}(t,f),\hat{f}(t,f))$\;
           $M(t,f)\leftarrow  (\lfloor t/r+\frac{1}{2}\rfloor,\lfloor f/q +\frac{1}{2}\rfloor)$\;
           
         }
      
    \While{$U$ not empty}{
      $(t',f')\leftarrow $ pop $ U$\;
      $M(t',f') = M(t,f)$\;
      
      }
    }
}}
  
\caption{Find Fixed Points}
\label{alg:basic-fixed-points}
\end{algorithm}
The basic idea is to start from each location of the time-frequency array and go in the direction
indicated by the function until you hit a location that you have already visited which is
declared the mode.  Each
node of the time-frequency representation is then marked by the end point of the path the
are on.  We include quantization factors as we may only want a coarse sampling of the time frequency plane for locations.  The number of locations in the fixed point map $M$ will generally be
much smaller than the total number of locations and thus we get a binary map indicating 
which locations are fixed points.  The key loop invariant that the algorithm maintains is that
by the end of the main while block every entry of $M$ either points to $(-1,-1)$ or to a mode.

\subsection{Empirical Evaluation of Simple Fixed Point Finder}

In our empirical evaluation we are primarily testing whether the fixed points found
by the algorithm correspond to the type of strucures we think are important in the
spectrogram.

\bibliographystyle{plain}
\bibliography{../bibliography}
\end{document}

%% This file is part of spectral_features.

%%     spectral_features is free software: you can redistribute it and/or modify
%%     it under the terms of the GNU General Public License as published by
%%     the Free Software Foundation, either version 3 of the License, or
%%     (at your option) any later version.

%%     spectral_features is distributed in the hope that it will be useful,
%%     but WITHOUT ANY WARRANTY; without even the implied warranty of
%%     MERCHANTABILITY or FITNESS FOR A PARTICULAR PURPOSE.  See the
%%     GNU General Public License for more details.

%%     You should have received a copy of the GNU General Public License
%%     along with spectral_features.  If not, see <http://www.gnu.org/licenses/>.


\documentclass[english]{article}
\usepackage[T1]{fontenc}
\usepackage[latin9]{inputenc}
\usepackage{listings}
\usepackage{graphicx}
\usepackage{esint}
\usepackage{babel}
\usepackage{fullpage}
\usepackage{hyperref}
\usepackage{amsmath}
\usepackage{amssymb}
\usepackage{subcaption}
\usepackage{tikz}
\usepackage{verbatim}

%\theoremstyle{plain}% default
\newtheorem{thm}{Theorem}[section]
\newtheorem{lem}[thm]{Lemma}
\providecommand*{\lemautorefname}{Lemma}
\newtheorem{prop}[thm]{Proposition}
\newtheorem{cor}[thm]{Corollary}
%\theoremstyle{definition}
\newtheorem{defn}{Definition}[section]
\newtheorem{conj}{Conjecture}[section]
\newtheorem{exmp}{Example}[section]
%\theoremstyle{remark}
\newtheorem{rem}{Remark}
\newtheorem{note}{Note}
\newtheorem{case}{Case}

\usepackage[ruled]{algorithm2e}
\renewcommand{\algorithmcfname}{ALGORITHM}
\SetAlFnt{\small}
\SetAlCapFnt{\small}
\SetAlCapNameFnt{\small}


\title{Edge Features for the Spectrogram}
\author{Mark Stoehr}
\date{\today}


\begin{document}

\maketitle
\tableofcontents

\section{Introduction}

We give a derivation of signal processing tools for extracting time-frequency
ridges. Generally the ridges we think of being in a time-frequency
display correspond either to the formants or the harmonics. In this
work we are primarily interested in formants as they are most relevant
for English language-speech recognition. An example sentence is in
Figure \ref{fig-sa1}
\begin{figure}
\centering{}\includegraphics[scale=0.5]{filters/multitaper_hermite_sa1}\caption{``She had your dark suit in greasy wash water all year''\label{fig-sa1}}
\end{figure}
our goal is to extract the formants which correspond to the ridges
in the spectrogram.

\section{Edge Criteria}

A signal processing tool used in \cite{amitobject,stoehrmasters} was to find maxima
along the directional derivatives in eight cardinal directions (as in \autoref{fig:binary-edge-orientations}) corresponding to two polarities
with four binary masks (as in \autoref{fig:binary-edge-masks}).
\begin{figure}
  \centering
  \includegraphics[height=2in]{/home/mark/Template-Speech-Recognition/Papers/plots/binary_edges.png}
  \caption{Eight binary edge masks}\label{fig:binary-edge-masks}
 \end{figure}
\begin{figure}
  \centering
  \includegraphics[width=3in]{/home/mark/Template-Speech-Recognition/Papers/plots/binary_edge_orientations.png}
  \caption{Eight binary edge orientations}\label{fig:binary-edge-orientations}
\end{figure}
The directionary derivative maxima were computed with coarse local distance computations but we may interpret
them as approximately computing time-frequency locations $(t,\omega)$ such that
\begin{equation}\label{eq:directional-derivative-defn}
g(t,\omega; v) =(\nabla \log |X(t,\omega)| )^\top v
\end{equation}
is greater than zero
and attains a maximum along the direction $v\in\mathbb{R}^2$ where \begin{equation}\tan^{-1}(v(2)/v(1)) = k\pi/4\end{equation}
where $k$ is an integer so that the angle of $v$ is an eighth root of unity. We may write these conditions as
\begin{align}\label{align:edge-conditions1}
g(t,\omega; v) &> 0\\
\label{align:edge-conditions2}(\nabla g(t,\omega; v))\cdot v &= 0\\
\label{align:edge-conditions3}\nabla ((\nabla g(t,\omega; v))\cdot v) \cdot v &< 0.
\end{align}
For the purposes of the following exposition we call $v$ the \textit{edge orientation}:
\begin{defn}
Let $S(t,f)$ be a signal over $(f,t)\subset \mathcal{F}\times \mathcal{T}$ then if $g(t,f; v)$ for $v\in\mathbb{R}^2$ is the directional
derivatve signal defined similarly to \autoref{eq:directional-derivative-defn} and if at a location $(t,f)$ all the conditions
\autoref{align:edge-conditions1},\autoref{align:edge-conditions2},
and \autoref{align:edge-conditions3} are met then $(t,f)$ is
said to have an edge of \textit{orientation} $v$.
\end{def}
Moreover, we note that if $\log |X(t,\omega)|$ has an edge of orientation $v$
at location $(t,\omega)$ then $v$ is approximately perpendicular
to the level curve of $\log |X|$
in a neighborhood of $(t,\omega)$.



\section{Partial Derivatives of the Spectrogram}

In order to apply \autoref{align:edge-conditions1} to spectrogram we will introduce some notation: let
$x(t)$ denote the speech signal and $X(t,\omega)$ denote its continuous short-time Fourier transform
\begin{equation}\label{eq:continuous-FT}
X(t,\omega) = \int x(\tau)h(t-\tau)e^{-i\omega \tau}\; d\tau
\end{equation}
where $h(t)$ is the windowing function for the short-time Fourier transform (usually it will be the
Hermite window).  Since we are generally working on the spectrogram $|X(t,\omega)|$ it is useful to
write the short-time Fourier transform in polar form where
\begin{equation}
X(t,\omega)= M(t,\omega)e^{i\phi(t,\omega)}
\end{equation}
where $M(t,\omega)=|X(t,\omega)|$.  In order to find locations $(t,\omega)$ that satisfy
the conditions in \autoref{align:edge-conditions1},\autoref{align:edge-conditions2},\autoref{align:edge-conditions3}
we need to compute the function $g(t,\omega ; v)$ which may be written
\begin{equation}
g(t,\omega ; v) = \frac{\partial X(t,\omega)}{\partial t} v(0) + \frac{\partial X(t,\omega)}{\partial \omega} v(1)
\end{equation} which means that we need to compute
the partial derivatives of the spectrogram. We may compute the partial derivatives using techniques
covered in  \cite{fitzfulop2009}
\begin{align*}
\frac{\partial}{\partial t} X(t,\omega) =& \frac{\partial}{\partial t} M(t,\omega) e^{i\phi(t,\omega)}\\
=& \frac{\partial M(t,\omega) }{\partial t} e^{i\phi(t,\omega)} + M(t,\omega)\cdot i \frac{\partial \phi(t,\omega) }{\partial t} e^{i\phi(t,\omega)}\\
=& \frac{\partial M(t,\omega) }{\partial t} e^{i\phi(t,\omega)} +  i \frac{\partial \phi(t,\omega) }{\partial t} X(t,\omega)
\end{align*}
which means that if $X^*(t,\omega)$ is the complex conjugate of $X(t,\omega)$ then
\begin{equation}
\frac{\partial}{\partial t} X(t,\omega) \cdot \frac{X^*(t,\omega)}{|X(t,\omega)|^2} = \frac{\partial M(t,\omega) }{\partial t}  \frac{1}{M(t,\omega)} +  i \frac{\partial \phi(t,\omega) }{\partial t}
\end{equation}
so that
\begin{equation}
\operatorname{Re}\left(\frac{\partial}{\partial t} X(t,\omega) \cdot \frac{X^*(t,\omega)}{|X(t,\omega)|^2}\right) = \frac{\partial M(t,\omega) }{\partial t}  \frac{1}{M(t,\omega)} = \frac{\partial \log |X(t,\omega)| }{\partial t}
\end{equation}
and using a similar argument we can show that
\begin{equation}\label{eq:realpartfreqderivative}
\operatorname{Re}\left(\frac{\partial}{\partial \omega} X(t,\omega) \cdot \frac{X^*(t,\omega)}{|X(t,\omega)|^2}\right) = \frac{\partial M(t,\omega) }{\partial \omega }  \frac{1}{M(t,\omega)} = \frac{\partial \log |X(t,\omega)| }{\partial \omega}
\end{equation}
so that we have
\begin{align}
g(t,\omega ; v) =& \operatorname{Re}\left(\nabla X(t,\omega) \cdot \frac{X^*(t,\omega)}{|X(t,\omega)|^2}\right) \cdot v\\
=& v(0)\operatorname{Re}\left(\frac{\partial X(t,\omega)}{\partial t} \cdot \frac{X^*(t,\omega)}{|X(t,\omega)|^2}\right)\cdots\\
& + v(1)\operatorname{Re}\left(\frac{\partial X(t,\omega)}{\partial \omega} \cdot \frac{X^*(t,\omega)}{|X(t,\omega)|^2}\right)
\end{align}
where $v=(v(0),v(1))^\top$. To compute the value of $g(t,\omega ; v)$ then we need to be able to compute the partial
derivatives of the short-time Fourier transform $X(t,\omega; h)$ (so that $h$ is the window of the transform)
\begin{align}
\frac{\partial}{\partial t} X(t,\omega; h) =& \frac{\partial}{\partial t} \int x(\tau) h(t-\tau) e^{-2\pi i \omega \tau}\; d\tau\\
=& \int x(\tau) h'(t-\tau) e^{-2\pi i \omega \tau}\; d\tau\\
\label{eq:dXdt}=& X(t,\omega ; \mathcal{D}h)
\end{align}
where $\mathcal{D}h(t)$ is simply the derivative window $h'(t)$ of the original window $h(t)$.  The derivative of $X(t,\omega ; h)$
with respect to frequency is
\begin{align}
\frac{\partial}{\partial \omega} X(t,\omega; h) =& \frac{\partial}{\partial \omega} \int x(\tau) h(t-\tau) e^{-2\pi i \omega \tau}\; d\tau\\
=& - 2\pi i \int x(\tau) \tau h(t-\tau) e^{-2\pi i \omega \tau}\; d\tau\\
=& -2 \pi itX(t,\omega ; h) + 2\pi itX(t,\omega ; h) - 2\pi i \int x(\tau) \tau h(t-\tau) e^{-2\pi i \omega \tau}\; d\tau\\
=&  -2 \pi itX(t,\omega ; h) + 2\pi i\int x(\tau)(t- \tau) h(t-\tau) e^{-2\pi i \omega \tau}\; d\tau\\
\label{eq:dXdw}=& 2\pi i ( X(t,\omega ; \mathcal{T}h) - tX(t,\omega ; h))
\end{align}
where $\mathcal{T}h(t) = t\cdot h(t)$ for $-\infty<t<\infty$.  Combining \autoref{eq:dXdt} with \autoref{eq:realpartfreqderivative}
to get 
\begin{equation}
\frac{\partial \log |X(t,\omega;\; h)|}{\partial t} = \operatorname{Re}\left(\frac{X(t,\omega ;\; \mathcal{D}h) X^*(t,\omega;\; h)}{|X(t,\omega;\; h)|^2}\right)
\end{equation}
and then by a similar argument we have
\begin{equation}\label{eq:frequency-edges-fast-form}
\frac{\partial \log |X(t,\omega;\; h)|}{\partial \omega} = -2\pi\operatorname{Im}\left(\frac{X(t,\omega ;\; \mathcal{T}h) X^*(t,\omega;\; h)}{|X(t,\omega;\; h)|^2}\right).
\end{equation}

\subsection{Empirical Plots of the magnitude derivatives}

We can look at some plots of these raw gradients in \autoref{fig:2HTS-256WS-dlogMdt-unsmoothed} and \autoref{fig:2HTS-256WS-dlogMdw-unsmoothed}
\begin{figure}
\centering
\includegraphics[scale=0.5]{exp/spectrogram_edges/plots/spectrogram_fcjf0sa1_edges_2HTS_256WS_dlogMdt.png}
\caption{Hermite window $\frac{\partial}{\partial t}\log M(t,\omega)$}
\label{fig:2HTS-256WS-dlogMdt-unsmoothed}
\end{figure}
\begin{figure}
\centering
\includegraphics[scale=0.5]{exp/spectrogram_edges/plots/spectrogram_fcjf0sa1_edges_2HTS_256WS_dlogMdw.png}
\caption{Hermite window $\frac{\partial}{\partial \omega}\log M(t,\omega)$}
\label{fig:2HTS-256WS-dlogMdw-unsmoothed}
\end{figure}
both of which are quite noisy we find that they are not very useful
for computing anything like the edges we normally use.


\section{Comparison of edges on different windows}

We provide a few comparisons of edges learned on two different windows



\section{Phase-Based Extraction}

The proposed method is based on the phase representation of speech
as in Figure \ref{fig-sa1-phase} 
\begin{figure}
\centering{}\includegraphics[scale=0.5]{filters/phase_multitaper_hermite_sa1}\caption{``She had your dark suit in greasy wash water all year''\label{fig-sa1-phase}}
\end{figure}
which is the same utterance as depicted in Figure \ref{fig-sa1}.
We will follow the presentation given in \cite{fitzfulop2009}. The
phase and magniutde are based on the discrete short-time Fourier transform
\begin{eqnarray*}
X(t,f;h) & = & \sum_{n=t}^{t+N-1}x(n)h(n-t)e^{-\frac{2\pi fn}{N}j}\\
 & = & e^{-\frac{2\pi ft}{N}j}\sum_{n=0}^{N-1}x(t+n)h(n)e^{-\frac{2\pi fn}{N}j}\\
 & = & e^{-\frac{2\pi ft}{N}}\hat{x}_{t}(k;h)
\end{eqnarray*}
so that the vector
\[
\left[\begin{array}{c}
\hat{x}_{t}(0;h)\\
\hat{x}_{t}(1;h)\\
\vdots\\
\hat{x}_{t}(N-1;h)
\end{array}\right]
\]
may be computed as the $N$-point fast Fourier transform of the digital
signal $x(t),x(t+1),\ldots,x(t+N-1)$ weighted by the window $h$.
Since $X(t,f)$ is a complex number we may write it as $M(t,f)e^{\phi(t,f)j}$
where $M(t,f)\geq0$ is the magnitude and $\phi(t,f)\in[0,2\pi)$
is the phase. By the definition of the complex logarithm
\[
\log X(t,f)=\log M(t,f)+\phi(t,f)j
\]
and the spectrogram in Figure \ref{fig-sa1} corresponds to $\log M(t,f)$
and Figure \ref{fig-sa1-phase} corresponds to $\phi(t,f)$ so they
are the real and complex parts respectively of $\log X(t,f)$. Figure
\ref{fig-sa1-phase} is difficult to interpret however its partial
derivative with respect to time contains important information about
the frequency information. Intuitively the phase of a pure tone should
change according to the frequency of that tone so an isolated tone
in a spectrogram should have nearly constant phase time derivative.
As shown in \cite{fitzfulop2009} we can compute the derivative as
\[
\frac{\partial\phi(t,f)}{\partial t}=\operatorname{Im}\left\{ \frac{X(t,f;h')X^{*}(t,f;h)}{|X(t,f;h)|^{2}}\right\} 
\]
where $h'(t)$ is the derivative of $h(t)$. We use the Hermite window
which has closed form expressions for all derivatives \cite{xiaoflandrin}
and a multitaper estimate so that we use windows $h_{0}(t),h_{1}(t),\ldots,h_{K-1}(t)$
and
\[
M(t,f;h,K)=\frac{1}{K}\sum_{k=0}^{K-1}M(t,f;h_{k}).
\]
 They also use a multitaper phase derivative estimate so that
\begin{equation}
\frac{\partial\phi(t,f;h_{k})}{\partial t}=\operatorname{Im}\left\{ \frac{X(t,f;\mathcal{D}h_{k})X^{*}(t,f;h_{k})}{|X(t,f;h_{k})|^{2}}\right\} \label{eq:dphi_dt}
\end{equation}
and
\[
\frac{\partial\phi(t,f;h,K)}{\partial t}=\frac{1}{K}\sum_{k=0}^{K-1}\frac{\partial\phi(t,f;h_{k})}{\partial t}.
\]
Multitaper estimates of the magnitude can significantly reduce the
variance without introducing much bias as long as the different tapers
are orthogonal \cite{percivalwalden93}. The time derivative 
\begin{figure}
\begin{centering}
\label{fig-sa1_dphi_dt}\includegraphics[scale=0.5]{filters/multitaper_hermite_dphi_dt_sa1}
\par\end{centering}

\caption{Phase partial derivative with respect to time plot}
\end{figure}
 can be seen in \autoref{fig-sa1_dphi_dt} which clear shows the
formnt contours apparent in \autoref{fig-sa1}. The derivative
of the phase with respect to frequency indicates information about
impulsive noise, however the information is not as clear as in this
picture. We note that 
\[
\frac{\partial\phi(t,f;h,K)}{\partial t}
\]
actually measures the phase derivative $\hat{x}_{t}(k)$ minus $\frac{2\pi k}{N}$
due to the definition of our short-time Fourier transform used above:
thus it measures the change in the phase. 

In \autoref{fig-sa1_dphi_dt} there are very sharp edges following
the formant contours. We can see this by filtering the display
\[
\frac{\partial\phi(t,f;h,K)}{\partial t}
\]
and computing a smoothed mixed derivative
\[
\frac{\partial^{2}\phi(t,f;h,K)}{\partial t\partial f}*g(t,f;\sigma)
\]
which may be computed using a formula similar to \autoref{eq:dphi_dt}
and convolving it with a Gaussian kernel, however that does not lead
to clean results. Instead, we use 
\begin{equation}\label{eq:phase-peak-phi}
\Phi(t,f;h,K,\sigma)=\frac{\partial\phi(t,f;h,K)}{\partial t}*\frac{\partial g(t,f;\sigma)}{\partial f}
\end{equation}
which is equivalent, theoretically. We obtain Figure \ref{fig-sa1_d2phi_dtdw_gauss}
by thresholding the values of $\Phi(t,f;h,K,\sigma)$.
\begin{figure}
\begin{centering}
\label{fig-sa1_d2phi_dtdw_gauss}\includegraphics[scale=0.5]{filters/multitaper_hermite_dphi_dt_thresh_diff_sa1}
\par\end{centering}

\caption{Gaussian difference applied to the phase partial derivative with respect
time}
\end{figure}
The values computed in Figure \ref{fig-sa1_d2phi_dtdw_gauss} track
the formants quite well as can be seen in 
\begin{figure}
\begin{centering}
\label{fig:sa1_d2phi_dtdw_gauss_overlay}\includegraphics[scale=0.5]{filters/multitaper_hermite_d2phi_dtdw_thresh_diff_overlayed_sa1}
\par\end{centering}

\caption{Gaussian difference applied to the phase partial derivative with respect
time layed on the spectrogram}
\end{figure}
 Figure \ref{fig:sa1_d2phi_dtdw_gauss_overlay}. We note that the
these frequency values depend only on the phase of the spectrogram
and not on the magnitude. One point to notice about the definition
given is that these points are found purely through the phase, however,
they naturally correspond to the high-energy regions of the spectrogram.
Thus, these curves may provide a natural normalization method.

\subsection{Spectral-Peak Finding}
\label{sec:spectral-peak-finding}
The figure in 
\autoref{fig:sa1_d2phi_dtdw_gauss_overlay} is computed by creating a binary spectral-peak map:
\begin{equation}
Y(t,f; \tau) =\begin{cases}
1 & \Phi(t,f;h,K,\sigma) > \tau\\
0 & \text{otherwise}
\end{cases}
\end{equation}
where $\Phi(t,f;h,K,\sigma)$ is computed by \autoref{eq:phase-peak-phi}.  We see that $Y(t,f)$ 
appears to track the formants but the output is not very useful since it 
gives a cluster of outputs around a spectral peak.  To get a single estimate we use a mean-shift-style
algorithm: \autoref{alg:mean-shift-peak} to cluster the peaks.
\begin{algorithm}[t]
\SetAlgoNoLine
\KwIn{Binary Phase Features $Y\in\{0,1\}^{T\times F}$, bandwidth parameter $\sigma > 0$, quantization factor $k\in\mathbb{Z}_+$, 
tolerance $\epsilon$ }
\KwOut{Clustered Peak Map $Z\in\{0,1\}^{T\times F}$}
  $U\in [T]\times [F]\leftarrow $ non-zero indices of $Y$\;
\For{$t=1,2,\ldots,T$}{
  $X \leftarrow \frac{1}{K}\begin{bmatrix} 1 & 2 & \cdots & kF\end{bmatrix}^\top$\;
  NotConverged $\leftarrow$ True\;
  \While{ NotConverged }{
    \For{$i=1,2,\ldots,kF$}{
      $D(i) = \frac{\sum_{u\in U}u(1)\exp\left\{-\frac{1}{2\sigma^2}\left[(u(0)-t)^2+(u(1)-X(i))^2\right]\right\}}{\sum_{u\in U}\exp\left\{-\frac{1}{2\sigma^2}\left[(u(0)-t)^2+(u(1)-X(i))^2\right]\right\}}$\;
      }
    NotConverged $\leftarrow \|X-D\| >\epsilon$\;
    $X\leftarrow D$\;
  }
  $X\leftarrow \newoperatorname{ceil}(X)$ \;
  $Z(t,X)\leftarrow 1$\;
}

\caption{Cluster Spectral Peak Map}
\label{alg:mean-shift-peak}
\end{algorithm}
some typical plots of the clustered output are pictured
\begin{figure}
\begin{subfigure}
\includegraphics[scale=0.75]{exp/binary_phase_meanshift/plots/mean_shift_binary_phase_peaks_train_dr7_mclk0_sa2_0.png}
\end{subfigure}
~
\begin{subfigure}
\includegraphics[scale=0.75]{exp/binary_phase_meanshift/plots/mean_shift_binary_phase_peaks_fcjf0_sa1_0.png}
\end{subfigure}

~
\begin{subfigure}
\includegraphics[scale=0.75]{exp/binary_phase_meanshift/plots/mean_shift_binary_phase_peaks_train_dr6_mdrd0_sx302_0.png}
\end{subfigure}

\caption{Clustered Phase Peak Output}
\label{fig:clustered-phase-peaks}
\end{figure}
in \autoref{fig:clustered-phase-peaks}.

\section{Reassignment Vector Field Clustering}

In analogy to \autoref{sec:spectral-peak-finding} we wish to produce a clustering based
on reassignment operators on the spectrogram.  A reassignment operator is a function
that takes a non-negative time-frequency representation $W$, a pair of coordinates $(t,f)$, and maps them to a new
set of coordinates $(t',f')$.  A common choice is to use a Wigner-Ville distribution $W$  One of the ways of formulating the reassignment
operators is as
\begin{equation}

\end{equation}

\section{Joint Phase and Magnitude Binary Features}

The normalization method is covered in the file 
\begin{lstlisting}
filters/ridge_computations.py
\end{lstlisting}
which has the code to reproduce the figures contained in this section.

We use the standard scale-space derivation but apply it to spectrograms
namely we wish to compute

\[
f_{w}(t,w;\sigma)=\frac{\partial}{\partial w}(g(t,w;\sigma)*\log X(t,w))
\]
where $\log$ denotes the complex logarithm and $g(t,w;\sigma)$ is
a normalized Gaussian at scale $\sigma$. We then get
\begin{eqnarray*}
f_{w}(t,w;\sigma) & = & \int dg_{w}(t-\tau,w-\omega;\sigma)\log|X(\tau,\omega)|d\tau d\omega\\
 &  & +j\int dg_{w}(t-\tau,w-\omega;\sigma)\phi(\tau,\omega)d\tau d\omega
\end{eqnarray*}
where we have used the fact that for $X(t,w)\neq0$ 
\begin{eqnarray*}
\log X(t,w) & = & \log|X(t,w)|e^{j\phi(t,w)}\\
 & = & \log|X(t,w)|+j\phi(t,w)
\end{eqnarray*}
and we let $dg_{w}=\frac{\partial}{\partial w}g$. We note that $f_{w}$
is the sum of two convolutions:
\begin{eqnarray*}
(dg_{w}*\log|X|)(t,w) & = & g_{w}(t,w;\sigma)*\frac{1}{|X(t,w)|}\frac{\partial}{\partial w}|X(t,w)|\\
 &  & \int g_{w}(t-\tau,w-\omega;\sigma)\frac{1}{|X(\tau,\omega)|}\frac{\partial}{\partial w}|X(\tau,\omega)|d\tau d\omega
\end{eqnarray*}
We are interested in
\[
\left(g*\frac{X_{th}X^{*}}{XX^{*}}\right)(t,w)=
\]

\section{Phase-guided Part Extraction}

We now combine the edge learning algorithms with the binary
phase feature algorithms to get a better collection of object
parts.  In the normal part-learning approach we get parts
that may not correspond to the significant features within
the spectrogram.  We extract axoids whose centerline has a
constant binary phase feature activation.

The patches have a length $L$ and a frequency span $\Omega$.
We will keep those all the same.  The main addition is that we only
extract parts along the contours defined by the binary ridge features.
Namely, given the list of formant point locations
$\{(t_i,f_i)\}$ we extract patches
\begin{equation}
Q(t_i,f_i) = [t_i-T/2,f_i-\Omega/2] \times [t_i+T/2,f_i+\Omega/2] \cap \mathbb{Z}^2
\end{equation}
and then these $\mathcal{Q}=\{Q(t_i,f_i)\}_{i\in I}$ are used to train
a product of Bernoullis finite mixture model which can be used in
a parts-based coding step \cite{AmitBernstein}.


\section{Designed Formant Parts}

In order to assist in detection we consider parts that are specifically designed to capture
formant structures. To do this we consider a large number of the formant style parts.
We start with analyzing binary feature transforms of the features for the utterance considered
in the previous section.

The contributions in this section:
\begin{enumerate}
\item Visual analysis of spectrograms showing binary features overlayed
\item A visual template for the structure of formant parts
\item A formant part dictionary that parametrically involves rotation
\end{enumerate}


To do this we first consider the binary features overlayed the spectrogram for each of the eight
types of edge features so that we get plots comparable to the binary phase features. First we consider the eight edge orientations from \autoref{fig:binary-edge-masks} and \autoref{fig:binary-edge-orientations}
which correspond to directions in the rows of
\begin{equation}
V = \begin{bmatrix} \phantom{-}1 & \phantom{-}0 \\
-1 & \phantom{-}0\\
\phantom{-}0 & \phantom{-}1 \\
\phantom{-}0 & -1\\
\phantom{-}1 & \phantom{-}1\\
-1 & -1\\
 \phantom{-}1 & -1\\
 -1 & \phantom{-}1.
\end{bmatrix}
\end{equation}
We can then see these eight orientations overlayed on
the spectrogram in \autoref{fig:spectrogram-overlayed-edges}

\begin{figure}
\label{fig:spectrogram-overlayed-edges}
\centering
\begin{subfigure}
    \centering
    \includegraphics[scale=0.75]{plots/binary_features_on_spectrogram_fcjf0sa1_edges_0.png}
    \caption{$(1,0)$ edges}
    \label{fig:overlayed_edges-0}
    \end{subfigure} 
~
\begin{subfigure}
    \centering
    \includegraphics[scale=0.75]{plots/binary_features_on_spectrogram_fcjf0sa1_edges_1.png}
    \caption{$(-1,0)$ edges}
    \label{fig:overlayed_edges-1}
    \end{subfigure} 
~
\begin{subfigure}
    \centering
    \includegraphics[scale=0.75]{plots/binary_features_on_spectrogram_fcjf0sa1_edges_2.png}
    \caption{$(0,1)$ edges}
    \label{fig:overlayed_edges-2}
    \end{subfigure} 
~
\begin{subfigure}
    \centering
    \includegraphics[scale=0.75]{plots/binary_features_on_spectrogram_fcjf0sa1_edges_3.png}
    \caption{$(-1,0)$ edges}
    \label{fig:overlayed_edges-3}
    \end{subfigure} 
~
    \caption{The tuple $(t,f)$ indicates the coordinates of the edge
orientation vector in time-freqeuncy coordinates with time $t$ and frequency $f$}
\end{figure}

\begin{figure}
\centering
\begin{subfigure}
    \centering
    \includegraphics[scale=0.75]{plots/binary_features_on_spectrogram_fcjf0sa1_edges_4.png}
    \caption{$(1,1)$ edges}
    \label{fig:overlayed_edges-4}
    \end{subfigure} 
~
\begin{subfigure}
    \centering
    \includegraphics[scale=0.75]{plots/binary_features_on_spectrogram_fcjf0sa1_edges_5.png}
    \caption{$(-1,-1)$ edges}
    \label{fig:overlayed_edges-5}
    \end{subfigure} 
~
\begin{subfigure}
    \centering
    \includegraphics[scale=0.75]{plots/binary_features_on_spectrogram_fcjf0sa1_edges_6.png}
    \caption{$(1,-1)$ edges}
    \label{fig:overlayed_edges-6}
    \end{subfigure} 
~
\begin{subfigure}
    \centering
    \includegraphics[scale=0.75]{plots/binary_features_on_spectrogram_fcjf0sa1_edges_7.png}
    \caption{$(-1,1)$ edges}
    \label{fig:overlayed_edges-7}
    \end{subfigure} 
\caption{The tuple $(t,f)$ indicates the coordinates of the edge
orientation vector in time-freqeuncy coordinates with time $t$ and frequency $f$}
\end{figure}

We begin with a model of ridges over spectrograms as piecewise
linear curves so that we may write a ridge as a sequence of joints
 (or knots to use spline terminology) 
\begin{equation}\label{eq:formant-spline-knots}
(t_0,f_0),(t_1,f_1),\ldots,(t_K,f_K)
\end{equation}
where $t_0 <t_1<\cdots <t_K$, $(t_0,f_0)$ is the starting time-frequency location, and
$(t_K,f_K)$ is the ending time-frequency location.  The piecewise-linear
assumption is that the ridge is well-approximately by a piecewise-linear curve
formed from the $K$ line segments $\{((t_{k-1},f_{k-1}),(t_k,f_k))\}_{k=1}^K$.  In order to build these models then we have to 
\begin{enumerate}
\item construct a model for the individual segments
\item build a model for piecing them together.
\end{enumerate} 
For the first task we consider the use of axoids as discussed
in \cite{arias2005near} which are parallelograms with one axis
vertical or horizontal.  Since we wish to chain these across time
we use axoids with one axis parallel to the frequency axis and the
time-axis is allowed have some angle $\theta$ between it
and the true time axis.  A positive $\theta$ indicates a rising formant
and a negative $\theta$ indicates a falling formant under this
model.  Although \cite{arias2005near} used axoids to detect
Gaussian signals in Gaussian noise we are concerned with detecting
objects by using a variety of binary features whose detection
provides evidence for the existence of a ridge. Thus, we formulate
our axoids as a product of Bernoullis model over a small patch
of the binary edge map produced from a spectrogram.
To piece these different axoids together we opt to use algorithms
similar to those considered in \cite{candes2006detecting},
however we also use explicit probability constraints.

\subsection{Segment Axoids}

The axoid that sits over a formant should have its centerline
approximately align with the center of the formant so that on
the spectrogram the centerline of the axoid will have large values
and small values towards the edge.  Let us formalize this
intuition to make it precise.  A formant under the model for
the previous section is segments joint at a sequence of knots
as in \autoref{eq:formant-spline-knots}.  So the centerline
of an axoid associated with $((t_{k-1},f_{k-1}),(t_k,f_k))$
and denote this axoid by $A$.  We call $((t_{k-1},f_{k-1}),(t_k,f_k))$
the \textit{central axis} of $A$ since it denotes the line along
which we hypothesize the formant runs.  The set of points in the
centerline $C(A)$ may be written
\begin{equation}\label{eq:centerline-axoid-def}
C(A) = \left\{ \begin{bmatrix}(1-\lambda) t_{k-1} + \lambda t_k\\ 
(1-\lambda) f_{k-1} + \lambda f_k\end{bmatrix}\mid 0\leq \lambda\leq 1\right\}.
\end{equation}
For every $t$ such that $t_{k-1}\leq t\leq t_{k}$ there is a unique $f$
such that $(t,f)$ is on $C(A)$. Let 
\begin{equation}
\lambda = \frac{t-t_{k-1}}{t_k-t_{k-1}}\in [0,1]
\end{equation}
since $t_{k-1}\leq t\leq t_k$ which means that
$(1-\lambda )f_{k-1} + \lambda f_k$ is in $[f_{k-1},f_k]$
and clearly 
\begin{equation}
(t,(1-\lambda )f_{k-1} + \lambda f_k) \in C(A)
\end{equation}
so we can define this as the $f$.  Any other $f'\neq f$
would be such that $(1-\lambda)f_{k-1}+\lambda f_k\neq f'$ thus 
meaning $(t,f)\not\in C(A)$.

We say that a point lies $(t,f)$ lies above the centerline if
\begin{equation}
f > (1-\lambda )f_{k-1} + \lambda f_k
\end{equation}
where
\begin{equation}
\lambda = \frac{t-t_{k-1}}{t_k-t_{k-1}}
\end{equation}
and we say that it lies below the centerline if
\begin{equation}
f < (1-\lambda )f_{k-1} + \lambda f_k.
\end{equation}
Axoid $A$ is said to have frequency span $\Omega$ and centerline
$((t_{k-1},f_{k-1}),(t_k,f_k))$ if
\begin{equation}\label{eq:axoid-set-definition}
A = \{ (t,f)\mid t_{k-1}\leq t\leq t_k,\; -\Omega/2\leq f - f_{k-1} - \frac{t-t_{k-1}}{t_k-t_{k-1}}(f_k-f_{k-1})\leq \Omega/2\}
\end{equation}
and this can be visualized in Figure~\ref{fig:axoid-basic-picture}.
\begin{figure}
\centering
\begin{tikzpicture}[scale=1.5]
    % Draw axes
    \draw [<->,thick] (0,3) node (yaxis) [above] {$y$}
        |- (5,0) node (xaxis) [right] {$x$};
    % Draw two intersecting lines
    \draw[dashed] (1,1.5) coordinate (a_1) -- (4,2) coordinate (a_2);
    \node [above, rotate=9.4623222] at (2.5,1.75) {$C(A)$};

    

    \draw (1,.75) coordinate (b_1) -- (1,2.25) coordinate (b_2);
    \draw (1,.75) coordinate (b_1) -- (4,1.25) coordinate (b_4);
    \draw (1,2.25) coordinate (b_2) -- (4,2.75) coordinate (b_3);
    \draw (4,2.75) coordinate (b_3) -- (4,1.25) coordinate (b_4);

    \node [left] at (0,.75) {$f_{k-1}-\frac{\Omega}{2}$};
    \draw[fill] (0,.75) circle [radius=0.025];
    \node [left] at (0,2.25) {$f_{k-1}+\frac{\Omega}{2}$};
    \draw[fill] (0,2.25) circle [radius=0.025];
    \node [left] at (0,1.5) {$f_{k-1}$};
    \draw[fill] (0,1.5) circle [radius=0.025];
    \node [below] at (1,0) {$t_{k-1}$};
    \draw[fill] (1,0) circle [radius=0.025];
    \node [below] at (4,0) {$t_k$};
    \draw[fill] (4,0) circle [radius=0.025];
    %% \draw (0,1.5) coordinate (b_1) -- (2.5,0) coordinate (b_2);
    %% % Calculate the intersection of the lines a_1 -- a_2 and b_1 -- b_2
    %% % and store the coordinate in c.
    %% \coordinate (c) at (intersection of a_1--a_2 and b_1--b_2);
    %% % Draw lines indicating intersection with y and x axis. Here we use
    %% % the perpendicular coordinate system
    %% \draw[dashed] (yaxis |- c) node[left] {$y'$}
    %%     -| (xaxis -| c) node[below] {$x'$};
    %% % Draw a dot to indicate intersection point
    %% \fill[red] (c) circle (2pt);
\end{tikzpicture}
\caption{Axoid over $((t_{k-1},f_{k-1}),(t_k,f_k))$ with frequency span $\Omega$}
\label{fig:axoid-basic-picture}

\end{figure}

Given that the axoid is supposed to sit over a ridge we expect
to find edges that are oriented towards the centerline. For a given
point $(t,f)\in A$ define a function $\theta(t,f)\in [0,2\pi)$ which
is the direction of the closest point in $C(A)$ from $(t,f)$.  We
let this quantity be $0$ if $(t,f)\in C(A)$. If
$(t,f)$ is above centerline $C(A)=((t_{k-1},f_{k-1}),(t_k,f_k))$
then the angle directly pointing towards the centerline is
\begin{equation}\label{eq:theta-above}
\theta(t,f) = \tan^{-1}\frac{f_k-f_{k-1}}{t_k-t_{k-1}} -\frac{\pi}{2} \mod 2\pi
\end{equation}
and if $(t,f)\in A$ is below then the angle is 
\begin{equation}\label{eq:theta-below}
\theta(t,f) = \tan^{-1} \frac{f_k-f_{k-1}}{t_k-t_{k-1}} + \frac{\pi}{2} \mod 2\pi
\end{equation}
and the proof of these statements is from simple geometry.  
Combining \autoref{eq:theta-above}, \autoref{eq:theta-below},
and \autoref{eq:axoid-set-definition} we get
\begin{equation}
\theta(t,f) = \begin{cases}
  \tan^{-1} \frac{f_k-f_{k-1}}{t_k-t_{k-1}} -\frac{\pi}{2} \mod 2\pi & f > f_{k-1} + \frac{t-t_{k-1}}{t_k-t_{k-1}}(f_k-f_{k-1})\\
\tan^{-1} \frac{f_k-f_{k-1}}{t_k-t_{k-1}} + \frac{\pi}{2} \mod 2\pi & f < f_{k-1} + \frac{t-t_{k-1}}{t_k-t_{k-1}}(f_k-f_{k-1})\\
0 & f = f_{k-1} + \frac{t-t_{k-1}}{t_k-t_{k-1}}(f_k-f_{k-1}).
\end{cases}
\end{equation}
We then expect to find edges with orientation $\theta'$ at $(t,f)$
if $\cos^{-1}(\theta' - \theta(t,f)) >0$.  We show what this means
in the context of a theoretical rising formant in
\autoref{fig:rising-formant-edge-theory} or a falling formant
in \autoref{fig:falling-formant-edge-theory}.
\begin{figure}
\centering
\input{plots/axoid_edges1}
\caption{Edge orientations used by rising formants}
\label{fig:rising-formant-edge-theory}
\end{figure}
\begin{figure}
\centering
\input{plots/axoid_edges2}
\caption{Edge orientations used by falling formants}
\label{fig:falling-formant-edge-theory}
\end{figure}
This means that we can now construct probability templates
for observing edges of a given rotation $\theta'$ at location
$(t,f)$.  Namely if we let $p(t,f,r)$ be a probability array
over time-frequency locations and edge orientations then
\begin{equation}\label{eq:oriented-probability-array-axoid}
p(t,f,r) = \begin{cases}
p_{obj} & \cos^{-1}|\theta' - \theta(t,f)| >\epsilon\\
p_{bgd} & \text{otherwise}
\end{cases}
\end{equation}
where $\epsilon > 0$ is small, $p_{obj}$ is large (close to one),
while $p_{bgd}$ is small and closer to zero.

\subsection{Discrete Segment Axoids}

The preceding discussion neglected the fact that these axoid
templates are over a discrete grid.  We now consider approaches
to parameterizing discrete axoids $A\in\mathcal{D}(A)$.  Just
as before, these are parameterized with their center lines $C(A)$
and a frequency span $\Omega$ except that due to the discrete
nature it is most natural to define $\Omega\in\mathbb{Z}$
where $\Omega=2F$ so that there are $F=\Omega/2$ frequency locations
above and below the center line $C(A)$.  The locations above
the same probability array as given by \autoref{eq:oriented-probability-array-axoid}
while the locations below the center line will share a probability
array at each location with the opposite polarities.
To define the center line $C(A)$ we may make use of discrete geometry
\cite{Dorst86}
however, for simplicity we initially just consider five possibilities:
\begin{enumerate}
\item rising: $(1,1,1,1)$, slope is $1$
\item slowly rising $(1,0,1,0)$, slope is $1/2$
\item constant $(0,0,0,0)$, slope is $0$
\item slowly falling $(-1,0,-1,0)$, slope is $-1/2$
\item falling $(-1,-1,-1,-1)$, slope is $-1$
\end{enumerate}
where the sequence specifies at each grid point whether there
is an increase, decrease, or level step.  The five different
digital centerlines are visualized in \autoref{fig:digital-axoid-centerlines-five}.
\begin{figure}
\centering
\begin{subfigure}
\begin{tikzpicture}
\draw[step=.5cm,gray,very thin] (0,-2) grid (2.5,2.5);
\fill[blue!40!white] (0,0) rectangle (.5,.5);
\fill[blue!40!white] (.5,.5) rectangle (1,1);
\fill[blue!40!white] (1,1) rectangle (1.5,1.5);
\fill[blue!40!white] (1.5,1.5) rectangle (2,2);
\fill[blue!40!white] (2,2) rectangle (2.5,2.5);
\end{tikzpicture}
\end{subfigure}
~
\begin{subfigure}
\begin{tikzpicture}
\draw[step=.5cm,gray,very thin] (0,-2) grid (2.5,2.5);
\fill[blue!40!white] (0,0) rectangle (.5,.5);
\fill[blue!40!white] (.5,.5) rectangle (1,1);
\fill[blue!40!white] (1,.5) rectangle (1.5,1);
\fill[blue!40!white] (1.5,1) rectangle (2,1.5);
\fill[blue!40!white] (2,1) rectangle (2.5,1.5);
\end{tikzpicture}
\end{subfigure}
~
\begin{subfigure}
\begin{tikzpicture}
\draw[step=.5cm,gray,very thin] (0,-2) grid (2.5,2.5);
\fill[blue!40!white] (0,0) rectangle (.5,.5);
\fill[blue!40!white] (.5,0) rectangle (1,.5);
\fill[blue!40!white] (1,0) rectangle (1.5,.5);
\fill[blue!40!white] (1.5,0) rectangle (2,.5);
\fill[blue!40!white] (2,0) rectangle (2.5,.5);
\end{tikzpicture}
\end{subfigure}
~
\begin{subfigure}
\begin{tikzpicture}
\draw[step=.5cm,gray,very thin] (0,-2) grid (2.5,2.5);
\fill[blue!40!white] (0,.5) rectangle (.5,0);
\fill[blue!40!white] (.5,0) rectangle (1,-.5);
\fill[blue!40!white] (1,0) rectangle (1.5,-.5);
\fill[blue!40!white] (1.5,-.5) rectangle (2,-1);
\fill[blue!40!white] (2,-.5) rectangle (2.5,-1);
\end{tikzpicture}
\end{subfigure}
~
\begin{subfigure}
\begin{tikzpicture}
\draw[step=.5cm,gray,very thin] (0,-2) grid (2.5,2.5);
\fill[blue!40!white] (0,.5) rectangle (.5,0);
\fill[blue!40!white] (.5,0) rectangle (1,-.5);
\fill[blue!40!white] (1,-.5) rectangle (1.5,-1);
\fill[blue!40!white] (1.5,-1) rectangle (2,-1.5);
\fill[blue!40!white] (2,-1.5) rectangle (2.5,-2);
\end{tikzpicture}
\end{subfigure}
\caption{Different digital center lines for the axoids}
\label{fig:digital-axoid-centerlines-five}
\end{figure}
We may also visualize where each of the eight edge orientations
pick up on a synthetic formant in \autoref{fig:eight-binary-edges-on-synthetic-formants}
\begin{figure}
\centering
\begin{subfigure}
\centering
\includegraphics[scale=0.5]{plots/seq1111edge_grid.png}
\end{subfigure}
\quad
\begin{subfigure}
\centering
\includegraphics[scale=0.5]{plots/seq1010edge_grid.png}
\end{subfigure}
\quad
\begin{subfigure}
\centering
\includegraphics[scale=0.5]{plots/seq0000edge_grid.png}
\end{subfigure}
\quad
\begin{subfigure}
\centering
\includegraphics[scale=0.5]{plots/seq-10-10edge_grid.png}
\end{subfigure}
\quad
\begin{subfigure}
\centering
\includegraphics[scale=0.5]{plots/seq-1-1-1-1edge_grid.png}
\end{subfigure}
\label{fig:eight-binary-edges-on-synthetic-formants}
\caption{Edges of the eight orientations detected on synthetic formant images}
\end{figure}
and from those we see a clear pattern to which edges are detected
which then allows us to define the five templates we will use
for finding formants.

\appendix

\section{Axoid Code}

This section discusses the computer implementation of axoids.
We have a program \texttt{src/axoids.py} which is the main
file for working with axoids.  We have implemented the following
functions:
\begin{enumerate}
\item \texttt{read\_axoid}
\item \texttt{write\_axoid}
\item \texttt{construct\_axoid}
\end{enumerate}
The data structure used to implement axoids is numpy records.

\bibliographystyle{plain}
\bibliography{../bibliography}
\end{document}
